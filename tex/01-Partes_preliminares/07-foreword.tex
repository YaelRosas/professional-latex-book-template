
\vspace*{2cm}

% Ejemplos 
{\Huge \textbf{Prólogo}} \\[1.0cm]
 
\begin{itemize}
    \item Un breve resumen del tema principal del libro y su importancia.
    \item Una descripción de cómo este libro se relaciona con el contexto actual o con el campo al que pertenece.
    \item Opiniones personales del autor del prólogo sobre la calidad o utilidad del libro.
    \item Una introducción al autor del libro, destacando su experiencia, logros o motivos para escribir la obra.
\end{itemize}


\vspace{1em}

\textbf{Estructura sugerida para el prólogo:}
\begin{enumerate}
    \item **Introducción**: Presentar brevemente el libro y su importancia.
    \item **Contexto**: Explicar cómo el libro encaja en el campo o área del conocimiento.
    \item **Recomendación**: Resaltar por qué el lector debería continuar leyendo el libro.
    \item **Firma del autor del prólogo**: Agregar el nombre, título o cargo de la persona que lo escribe.
\end{enumerate}

Por ejemplo:

\textit{}
\textbf{\textit{"Cuando me pidieron que escribiera el prólogo de este libro, no tuve ninguna duda en aceptar. Conozco al autor desde hace años, y he sido testigo de su dedicación y esfuerzo por compartir el conocimiento de manera clara y precisa. Este libro no es solo un manual técnico; es un viaje por el fascinante mundo de la inteligencia artificial, presentado de forma accesible tanto para principiantes como para expertos. Espero que este texto inspire a muchos a continuar explorando y aprendiendo sobre esta disciplina tan emocionante."}}


 
\vspace{2em}
\hfill \textit{[Nombre del autor del prólogo]} \\
\hfill \textit{[Título o cargo]} \\
\hfill \textit{[Lugar y fecha]} \\
\clearpage