\documentclass{styles/my_class}         % Clase personalizada 
\usepackage{styles/custom-style}     % Cargar el estilo desde 'config/mi_estilo.sty'
 
%=== Glosario===
 % --- Definir glosario personalizado ---en caso de no usarlo puedes solo no imprimir el ejemplo comentandolo(Esta en la seccion material de referencia de este archivo)
 \newglossary[glgcon]{constantes}{glscon}{glocon}{Glosario Definido por el usuario}%(Si cambias el nombre tambien hazlo en las definiciones de glosarios /tex/03-Referencia_Configuracion/02-definicion_glosarios)
 %======================================================================
% === Glosario de términos generales (type: main) ===
%======================================================================
\newglossaryentry{algoritmo}{
  name={\textbf{alg}},
  plural={\textbf{algs}},
  description={conjunto de pasos definidos para resolver un problema}
}

\newglossaryentry{modelo}{
  name={\textbf{modelo}},
  plural={\textbf{modelos}},
  description={representación simplificada de un sistema o fenómeno}
}
%======================================================================
% === Glosario de acrónimos y siglas (type: acronym) ===
%======================================================================
\newacronym[longplural={Unidades Centrales de Procesamiento}]{cpu}{\textbf{CPU}}{Unidad Central de Procesamiento}
\newacronym[longplural={Unidades de Procesamiento Gráfico}]{gpu}{\textbf{GPU}}{Unidad de Procesamiento Gráfico}
%======================================================================
% === Glosario de símbolos (type: symbols) ===
%======================================================================
\newglossaryentry{alpha}{
  type=symbols,         
  name={\textbf{$\alpha$}},
  description={coeficiente de expansión térmica}
}

\newglossaryentry{lambda}{ 
  type=symbols,
  name={\textbf{$\lambda$}},
  description={longitud de onda}
}

%======================================================================
% === Definido por el usuario ===
%======================================================================
\newglossaryentry{planck}{
  type=constantes,
  name={\textbf{$h$}},
  description={constante de Planck: \(6.626 \times 10^{-34}~\text{J$\cdot$s}\)}
}

\newglossaryentry{gravitacional}{
  type=constantes,
  name={\textbf{$G$}},
  description={constante de gravitación universal: $6.674 \times 10^{-11}~\text{N·m}^2/\text{kg}^2$}
}
  % Carga las definiciones de glosario
 \makeglossaries

%=== Bibliografia===
 \addbibresource{tex/03-Referencia_Configuracion/01-references.bib} % Base de datos bibliograficos


\begin{document}
%=============================================================
%=== Partes preliminares
%=============================================================
 
\thispagestyle{empty} % Sin numeración en esta página
\let\cleardoublepage\clearpage % Para evitar la hoja en blanco que se genera al terminar en hoja impar 
 
\begin{titlepage}
  \thispagestyle{empty} % Sin numeración de página
  \AddToShipoutPictureBG*{\includegraphics[width=\paperwidth,height=\paperheight]{images/tex.png}}%
  \vspace*{\fill} % Para evitar solapamientos

\end{titlepage}
 
 \addcontentsline{toc}{chapter}{Acerca de la obra}

\clearpage

\begin{titlepage}
    \thispagestyle{empty} % Sin numeración de página
    \vfill

    
    \begin{center}
        
        \vspace*{2cm}
        \textcolor{black}{\Huge \textbf{Plantilla de Libro}} \\[0.7cm] 
        {\large Primera Edición} \\[0.3cm]  % Subtítulo

        % Línea decorativa
        \rule{\textwidth}{0.4pt} \\[1cm]

        {\Huge \textit{Yael R.G.}} \\[0.5cm]  % Autor 1
        {\normalsize Lic. M.A.C.} \\[1.5cm]  % Títulos academicos 1

        {\Huge \textit{Autor 2}} \\[0.5cm]  % Autor 2
        {\normalsize Títulos del Autor 2} \\[5cm]  % Títulos academicos 2
        \maketitle

        % Logotipo
        \includegraphics[width=3cm]{images/y.png} \\[1cm]  % Logotipo editorial

        % Información de la editorial
        {\Large Editorial: Autopublicado} \\[1cm]  % Editorial
        {\large C.D.M.X, 2025}  % Ciudad y año de publicación
            
    \end{center}
    \vfill
\end{titlepage}
\clearpage 
 %=======================
% Información Real de la Plantilla
%=======================
\thispagestyle{empty} % Sin numeración en esta página

\begin{center}
\vspace*{1cm}
\textcolor{black}{\Large \textbf{© 2025, Yael R.G.}} \\[0.3cm]

\textbf{Aviso:} El ISBN, depósito legal y los enlaces incluidos en esta plantilla son meramente ilustrativos. No corresponden a un libro real, ya que este proyecto tiene como único propósito servir como plantilla. \\[0.5cm]


Plantilla creada por Yael R.G. \\[0.3cm]
Créditos: \href{https://www.linkedin.com/in/yael-rg-digital}{LinkedIn: www.linkedin.com/in/yael-rg-digital}. \\[0.3cm]
Disponible en: \href{https://github.com/YaelRosas}{GitHub: github.com/YaelRosas} \\[0.5cm]

\textbf{Licencia:} Los derechos para el uso de esta plantilla son bajo una licencia Creative Commons Attribution 4.0 International License (CC BY 4.0). \\[0.5cm]

\begin{center}
    Puede encontrar un resumen de la licencia en: \\
    \url{https://creativecommons.org/licenses/by/4.0/} \\[0.5cm]
    
    El código legal completo está disponible en: \\
    \url{https://creativecommons.org/licenses/by/4.0/legalcode}
    \end{center}
    

\textbf{Usted es libre de:}
\end{center}

\begin{center}
\begin{tabular}{c}
\begin{minipage}{0.9\textwidth}
\begin{itemize}
    \item \textbf{Compartir} — copiar y redistribuir el material en cualquier medio o formato.
    \item \textbf{Adaptar} — remezclar, transformar y construir sobre el material para cualquier propósito, incluso comercialmente.
\end{itemize}
\end{minipage}
\end{tabular}
\end{center}

\begin{center}
\textbf{Bajo los siguientes términos:}
\end{center}

\begin{center}
\begin{tabular}{c}
\begin{minipage}{0.9\textwidth}
\begin{itemize}
    \item \textbf{Atribución} — Debe dar el crédito apropiado, proporcionar un enlace a la licencia e indicar si se han realizado cambios. Puede hacerlo de cualquier manera razonable, pero no de ninguna manera que sugiera que el licenciante lo respalda a usted o a su uso.
    \item \textbf{Sin restricciones adicionales} — No puede aplicar términos legales o medidas tecnológicas que restrinjan legalmente a otros a hacer cualquier cosa que permita la licencia.
\end{itemize}
\end{minipage}
\end{tabular}
\end{center}

\begin{center}
{\large Primera edición, 2025} \\[0.5cm]

{\large \textbf{Equipo de Edición:}} \\[0.5cm]
{\large Editor: Yael RG} \\[0.5cm]
{\large Técnico editorial: ChatGPT (OpenAI)} \\[1.0cm]

{\large \textbf{Equipo de Diseño:}} \\[0.5cm]
{\large Diseñador: Yael RG} \\[0.5cm]
{\large Técnico de diseño: ChatGPT (OpenAI)} \\[1.0cm]
\end{center}


\clearpage
\thispagestyle{empty} % Sin numeración en esta página

%=======================
% Información de ejemplo
%=======================
\begin{center}
    \vspace*{2cm}
{\large Ejemplo de copyright} \\[0.8cm]
\textcolor{black}{\Large \textbf{© 2025, Yael R.G.}} \\[0.3cm]
{\large Todos los derechos reservados.} \\[0.8cm]
{\large Ninguna parte de este libro puede ser reproducida, distribuida o transmitida de ninguna forma ni por ningún medio, sin el permiso previo del autor.} \\[0.5cm]
{\small \textit{El uso de este material está sujeto a las leyes de propiedad intelectual aplicables.}} \\[0.5cm]

{\large Primera edición, 2025} \\[0.5cm]
{\large ISBN: 123-456-7890} \\[0.5cm]
{\large Depósito legal: M-10289-2025} \\[1.5cm]

{\large \textbf{Equipo de Edición:}} \\[0.5cm]
{\large Editor: Nombre del Editor} \\[0.5cm]
{\large Técnico editorial: Nombre del Técnico} \\[1.0cm]

{\large \textbf{Equipo de Diseño:}} \\[0.5cm]
{\large Diseñador: Nombre del Diseñador} \\[0.5cm]
{\large Técnico de diseño: Nombre del Técnico} \\[1.0cm]

{\large \textbf{Equipo de Producción:}} \\[0.5cm]
{\large Director: Nombre del Director} \\[0.5cm]
{\large Coordinador: Nombre del Coordinador} \\[5.0cm]

\begin{textblock}{15}(0.5,13.0)
    \raggedright
    {\scriptsize \textit{Este libro incluye enlaces a sitios web cuya gestión, mantenimiento y control son responsabilidad exclusiva de terceros ajenos a la editorial y al autor. Los enlaces y otras referencias a sitios web se incluyen con fines estrictamente informativos y se proporcionan en el estado en que se encuentran en el momento de la publicación, sin garantías, expresas o implícitas, sobre la información que se proporcione en ellos. Los enlaces no implican el aval de la editorial o del autor a dichos sitios, páginas web, funcionalidades y sus respectivos contenidos, ni cualquier asociación con sus administradores. En consecuencia, la editorial y el autor no asumen responsabilidad alguna por los daños que puedan derivarse de posibles infracciones de los derechos de propiedad intelectual e industrial que puedan contener dichos sitios web, ni por las pérdidas, delitos, daños o perjuicios derivados, directa o indirectamente, del uso de dichos sitios web y su información. Al acceder a estos enlaces externos, el usuario estará sujeto a las políticas de protección de datos, privacidad, prácticas y otros contenidos de dichos sitios web, y no a los de la editorial o el autor de este libro.}} \\[0.1cm]
\end{textblock}
\end{center}

\vfill
\clearpage 
 
\thispagestyle{empty} % Sin numeración en esta página

\begin{center}
    \vspace*{2cm}

    % Ejemplos 
    {\Huge \textbf{Dedicatoria}} \\[1.5cm]
    {\large Una breve nota para reconocer y dedicar el trabajo, aunque no necesariamente contribuyentes directos en la creación.} \\[1.5cm]

    \vspace*{3cm}
    {\large \textbf{Nombre del Autor}} \\[1.0cm]
    {\large Ciudad, Año}

\end{center}

\vfill
\clearpage
 
 
\thispagestyle{empty} % Sin numeración de página


\vspace*{2cm}

% Ejemplos 
{\Huge \textbf{Agradecimientos}} \\[1.0cm]
 
Agradecimientos a personas que han apoyado o influido en el desarrollo:

\begin{center}
\begin{itemize}
    \item A \textbf{[Mentor, experto]}, por sus valiosos comentarios durante el desarrollo de este proyecto.
    \item A \textbf{[Equipo o colaboradores]}, cuyo trabajo fue esencial para completar este libro.
    \item A mi familia y amigos, por su paciencia, comprensión y aliento.
\end{itemize}
\end{center}
A todos ustedes, gracias por hacer que este proyecto sea una realidad.
\clearpage 
 \frontmatter % Inicia numeración romana
 \thispagestyle{empty}
\vspace*{2cm}

{
\Huge \textbf{\textcolor{black}{Epígrafe}} \\[1.0cm]

\epigraph{
\textcolor{black}{
Una cita breve, un verso o una frase que sugiera un tema, resuma una idea principal o establezca el tono general de la obra.
}
}{
\textcolor{black}{-- Autor}
}
}

\clearpage

 
\vspace*{2cm}

% Ejemplos 
{\Huge \textbf{Prólogo}} \\[1.0cm]
 
\begin{itemize}
    \item Un breve resumen del tema principal del libro y su importancia.
    \item Una descripción de cómo este libro se relaciona con el contexto actual o con el campo al que pertenece.
    \item Opiniones personales del autor del prólogo sobre la calidad o utilidad del libro.
    \item Una introducción al autor del libro, destacando su experiencia, logros o motivos para escribir la obra.
\end{itemize}


\vspace{1em}

\textbf{Estructura sugerida para el prólogo:}
\begin{enumerate}
    \item **Introducción**: Presentar brevemente el libro y su importancia.
    \item **Contexto**: Explicar cómo el libro encaja en el campo o área del conocimiento.
    \item **Recomendación**: Resaltar por qué el lector debería continuar leyendo el libro.
    \item **Firma del autor del prólogo**: Agregar el nombre, título o cargo de la persona que lo escribe.
\end{enumerate}

Por ejemplo:

\textit{}
\textbf{\textit{"Cuando me pidieron que escribiera el prólogo de este libro, no tuve ninguna duda en aceptar. Conozco al autor desde hace años, y he sido testigo de su dedicación y esfuerzo por compartir el conocimiento de manera clara y precisa. Este libro no es solo un manual técnico; es un viaje por el fascinante mundo de la inteligencia artificial, presentado de forma accesible tanto para principiantes como para expertos. Espero que este texto inspire a muchos a continuar explorando y aprendiendo sobre esta disciplina tan emocionante."}}


 
\vspace{2em}
\hfill \textit{[Nombre del autor del prólogo]} \\
\hfill \textit{[Título o cargo]} \\
\hfill \textit{[Lugar y fecha]} \\
\clearpage
 \vspace*{2cm}
\textcolor{black}{\Huge \textbf{Prefacio}} \\[0.7cm] % Título

El prefacio es una sección escrita por el autor principal del libro.
Su propósito es proporcionar al lector una visión general sobre las motivaciones para escribir la obra,
el enfoque adoptado, los objetivos que se esperan alcanzar y 
la audiencia a la que está dirigida. Además, en esta sección, 
el autor puede agradecer a las personas o instituciones que contribuyeron al desarrollo del trabajo.
El prefacio sirve como una introducción personal que conecta al autor con el lector antes de entrar
en el contenido principal del libro.


\textbf{Ejemplo de prefacio:}

Este libro surge de mi deseo de compartir el conocimiento adquirido durante años de estudio 
y práctica en el campo de la inteligencia artificial. La idea de crear esta obra nació de la 
necesidad de contar con un recurso integral que sirviera tanto a principiantes como a expertos para 
comprender los fundamentos y las aplicaciones más avanzadas de esta fascinante disciplina.
El camino para materializar este proyecto no habría sido posible sin el apoyo incondicional de mis 
colegas y estudiantes, quienes enriquecieron mis ideas con preguntas desafiantes y discusiones 
profundas. Agradezco especialmente a mis mentores y colaboradores por sus valiosas orientaciones, 
así como a mi familia, por su paciencia y apoyo constante durante el tiempo dedicado a la elaboración 
de este libro.
Espero que esta obra se convierta en una herramienta útil para todos aquellos interesados en explorar 
y desarrollar soluciones innovadoras en el ámbito de la inteligencia artificial. Mi objetivo es que 
cada lector encuentre aquí inspiración y recursos para enfrentar los desafíos y aprovechar las 
oportunidades que este campo ofrece.
--- *Nombre del autor*  
\clearpage
 \vspace*{2cm}
\textcolor{black}{\Huge \textbf{Tabla de Contribuciones}} \\[0.7cm] % Título

El presente libro ha sido enriquecido gracias a las valiosas contribuciones de diversas personas y entidades. A continuación, se describen sus aportes específicos:

\begin{itemize}
    \item \textbf{Prólogo:}  
    El prólogo ha sido escrito por la Dra. Ana Martínez, profesora e investigadora en inteligencia artificial. Su propósito es presentar el contenido de esta obra al lector, destacando su relevancia y calidad, así como los aportes significativos al área de aprendizaje automático.
    
    \item \textbf{Capítulo X:}  
    El capítulo sobre redes neuronales fue revisado y ampliado por el Ing. Juan Pérez, especialista en arquitecturas de redes profundas.

    \item \textbf{Revisión Técnica:}  
    La revisión técnica del manuscrito fue realizada por la organización TechAI, asegurando precisión y claridad en los conceptos presentados.

    \item \textbf{Edición y Formato:}  
    Agradecemos a Editorial Innovate por su trabajo excepcional en la edición y formato final del documento.
\end{itemize}
\clearpage
 \vspace*{2cm}
\textcolor{black}{\Huge \textbf{Nota del autor}} \\[0.7cm] % Título

Este libro fue escrito con la intención de compartir conocimientos adquiridos a lo largo de varios años de trabajo y estudio.\\

No pretende ser una obra definitiva, sino un punto de partida para el lector curioso, crítico y reflexivo.\\

Agradezco a quienes, directa o indirectamente, inspiraron este proyecto, y a quienes dedican su tiempo a leerlo.\\

Las opiniones expresadas aquí son personales y no representan necesariamente la postura de ninguna institución.
\clearpage
 \tableofcontents
 \clearpage
 \renewcommand{\listfigurename}{Lista de Figuras}
 \listoffigures
 \clearpage
 \renewcommand{\listtablename}{Lista de Tablas}
 \listoftables
 
 \mainmatter % Inicia numeración arábiga

%=============================================================
%=== Cuerpo del libro
%=============================================================
 \let\cleardoublepage\clearpage % Para evitar la hoja en blanco que se genera al terminar en hoja impar 
 \part{Contenido}
 \cleardoublepage
\chapter*{Introducción}
\addcontentsline{toc}{chapter}{Introducción}  % Añadir al índice manualmente

% 11-introduction.tex

La introducción de tu libro o proyecto debe proporcionar una visión general de los temas que se tratarán, los objetivos del proyecto y cualquier otra información relevante. 

En esta sección puedes incluir:

\begin{itemize}
    \item El propósito del libro o proyecto.
    \item Breve descripción de los temas que se abordarán.
    \item Objetivos y metas principales.
    \item Información relevante para el lector.
\end{itemize}

Puedes agregar más detalles y personalizar esta plantilla según sea necesario para tu proyecto.
 
 \pagenumbering{arabic} % Evita duplicados en numeración
\cleardoublepage

% ============================
% SECCIÓN: ESTRUCTURA GENERAL DE UN CAPÍTULO 
 %(sección 6 en config.tex)
%=================
\chapter{Capitulo}
\fancyhead{} % Borra el encabezado
\thispagestyle{fancy}

Texto...Texto...Texto...Texto...Texto...Texto...Texto...Texto...Texto...

Texto...Texto...Texto...Texto...Texto...Texto...Texto...Texto...Texto...
\section{Seccion} 
Texto...Texto...Texto...Texto...Texto...Texto...Texto...Texto...Texto...

Texto...Texto...Texto...Texto...Texto...Texto...Texto...Texto...Texto...
\subsection{Subseccion}
Texto...Texto...Texto...Texto...Texto...Texto...Texto...Texto...Texto...

Texto...Texto...Texto...Texto...Texto...Texto...Texto...Texto...Texto...
\subsubsection{Subsubsection}

\paragraph{Parrafo}
Texto...Texto...Texto...Texto...Texto...Texto...Texto...Texto...Texto...

Texto...Texto...Texto...Texto...Texto...Texto...Texto...Texto...Texto...
\subparagraph{Subparrafo}
Texto...Texto...Texto...Texto...Texto...Texto...Texto...Texto...Texto...

Texto...Texto...Texto...Texto...Texto...Texto...Texto...Texto...Texto...


 % ======== 
 % Solution
 \begin{activity}[Trabaja en parejas]
  Encuentra tres ejemplos de grupos que no sean abelianos.
\end{activity}

\begin{definition}[Este término es clave]
  Un \emph{grupo} es un conjunto con una operación binaria que satisface la asociatividad, tiene un elemento neutro y cada elemento posee un inverso.
\end{definition}

\begin{example}[Grupo simple]
  El conjunto de los enteros \( \mathbb{Z} \) con la suma es un grupo.
\end{example}

\begin{exercise}[Intenta resolverlo]
  Demuestra que el conjunto de los enteros pares forma un subgrupo de \( \mathbb{Z} \).
\end{exercise}

\begin{generality}[Por qué importa la generalidad]
  El concepto de grupo abstrae la idea de simetría, aplicable en álgebra, geometría y física.
\end{generality}

\begin{note}[Consejo sobre notación]
  El elemento neutro suele denotarse por \( e \) o \( 1 \), según el contexto.
\end{note}

\begin{property}[Clausura]
  El producto de dos elementos cualesquiera de un grupo pertenece también al grupo.
\end{property}

\begin{remark}[Confusión común]
  Recuerda: la operación del grupo no tiene que ser multiplicación.
\end{remark}

\begin{solution}[prueba de conteo]
  La derivada de \( f(x) = x^2 \) es \( f'(x) = 2x \).
  Aplicamos la regla de la potencia para obtener este resultado.
\end{solution}

\begin{summary}[Ideas clave del capítulo]
  Este capítulo abordó la definición formal de grupo, ejemplos importantes como \((\mathbb{Z}, +)\), y propiedades clave como la existencia del neutro e inverso.
\end{summary}

\begin{theorem}[Propiedad básica de los grupos]
  Si \( G \) es un grupo y \( a, b \in G \), entonces \( ab^{-1} \in G \).
\end{theorem}

\begin{warnbox}[Precaución al usar propiedades conmutativas]
  No todas las operaciones son conmutativas. Asegúrate de verificar las propiedades antes de aplicarlas.
\end{warnbox}
    
%===========================================================
 \section{Entornos predefinidos} 
 
  quote	      Sangrado especial
  quotation	  Similar a quote
  verse	      Para poesía (maneja saltos de línea)
  center	    Centra texto
  flushleft	  Alinea a la izquierda
  flushright	Alinea a la derecha
  itemize	    Lista con viñetas
  enumerate	  Lista numerada
  description	Lista con etiquetas personalizadas
 

\section{Jerarquía de secciones}

Los comandos de jerarquía de secciones son utiles para generar el índice y agregar "títulos". 

\begin{center}
  \begin{tabular}{|c|l|c|}
  \hline
  \textbf{Nivel} & \textbf{Etiqueta} & \textbf{Disponible en} \\
  \hline
  0 & \verb|\part| & \verb|book| o \verb|report| \\
  1 & \verb|\chapter| & \verb|book| o \verb|report| \\
  2 & \verb|\section| & Todas las clases \\
  3 & \verb|\subsection| & Todas las clases \\
  4 & \verb|\subsubsection| & Todas las clases \\
  5 & \verb|\paragraph| & Todas las clases \\
  6 & \verb|\subparagraph| & Todas las clases \\
  \hline
  \end{tabular}
  \end{center}

Sin embargo, \textbf{no están pensadas para ser usadas como simples contenedores de texto}.  
Si se desea escribir párrafos de texto normales, lo correcto es usar líneas en blanco entre bloques de texto en vez de estas.

\begin{custompar}
  Si se desea escribir párrafos con formato personalizado, lo recomendable es definir un entorno personalizado usando 
  \textbf{\texttt{\textbackslash newenvironment}} en el archivo de configuración (sección 6 en este proyecto,tambien se puede modificar el formato del texto general).
  \end{custompar}

% ============================
%  Alineaciones de párrafos: 
 
 \textbf{Alineaciones de párrafos}

 \begin{flushleft}
 Este párrafo estará alineado a la izquierda.
 \end{flushleft}
 
 \begin{center}
 Este párrafo estará centrado.
 \end{center}
 
 \begin{flushright}
 Este párrafo estará alineado a la derecha.
 \end{flushright}
 
% ============================
%  Texto con marcos 
 
\textbf{Texto con marcos}
 
 % Requiere: \usepackage{framed}
 \begin{framed}
 Este párrafo está dentro de un marco.
 \end{framed}
 
 % Requiere: \usepackage{mdframed}
 \begin{mdframed}
 Este párrafo está dentro de un marco con más opciones de estilo.
 \end{mdframed}
 
 \clearpage
% ============================
% SECCIÓN: EJEMPLO LISTAS
 %(seccion 14 en config.tex)
 
 \section{Ejemplo de listas}
 
  \subsection{Lista estandar}
   Ejemplo \texttt{Titulo de la lista}:


  
  \subsection{Lista con titulos definidos}
   \begin{description}
      \item[Contexto:] Comprender el contexto histórico del tema.
      \item[Problema:] Identificar los problemas clave a resolver.
      \item[Revisión:] Revisar trabajos previos relevantes.
    \end{description}
    
    \clearpage
% ============================
% SECCIÓN: EJEMPLO DE CITAS Y BIBLIOGRAFIA
 % (seccion 12 en config.tex)
 % Requisitos (ya estan implementados):
 %  
 % 1-Cargar el archivo .bib en  main.tex (Justo despues de \documentclass) 
 % 2-Definir la cita en el archivo .bib (En /tex/Material_de_referencia)
 % 3-Imprimir la bibliografia con \printbibliography (Esta al final del main.tex)
 \section{Ejemplo de citas}
 
 % Cómo citar en LaTeX (bibliografía .bib)

 Este es un ejemplo de cita de un libro: \cite{EjemploLibro}.  
 Se usa para referirse a libros, artículos, documentación, etc.
 
 % Archivo .bib obligatorio
 Las citas deben estar definidas en un archivo \textbf{bib.}  
 En este proyecto
 
 % Mostrar bibliografía al final
 El comando \texttt{\textbf{printbibliography}} imprime solo las fuentes citadas en el documento.  
Se coloca normalmente en las últimas partes del documento, como en \texttt{\textbf{main.tex}}.

El comando \texttt{\textbf{nocite\{*\}}} fuerza a incluir todas las entradas del archivo \texttt{\textbf{.bib}}, incluso si no fueron citadas.
 % Ejemplo de cita textual corta
 \begin{quote}
 Esta es una cita corta. Se muestra con márgenes más estrechos.
 \end{quote}
 
 % Ejemplo de cita textual larga
 \begin{quotation}
 Esta es una cita más larga.
 
 Puede contener varios párrafos.
 \end{quotation}
 \clearpage
% ============================
%  SECCIÓN: EJEMPLO NOTAS AL PIE Y REFERENCIAS
 %(seccion 12 en config.tex)
 
 \section{Ejemplo de notas al pie de página y referencias cruzadas}

 Este ejemplo muestra cómo crear referencias cruzadas y notas al pie de página en \LaTeX.
 
 Para crear una referencia cruzada, utiliza el comando \textbf{\texttt{\textbackslash label\{etiqueta\}}}
 para asignar una etiqueta a una sección. Luego, usa \textbf{\texttt{\textbackslash ref\{etiqueta\}}} o 
 \textbf{\texttt{\textbackslash autoref\{etiqueta\}}} para referenciarla en otra parte del documento.
 
 Para agregar una nota al pie, utiliza \textbf{\texttt{\textbackslash footnote\{texto de la nota\}}}.
 
 

\footnote{Coloca el comando justo donde deseas que aparezca el número de la nota al pie.}
 
 \subsection{Sección de ejemplo} 

 \label{sec:ejemplo}% Lleva a lo que este abajo de la etiqueta
 Esta sección de ejemplo es referenciada con \texttt{\textbackslash label\{etiqueta\}}.


 Texto...Texto...Texto...Texto...Texto...Texto...Texto...Texto...Texto...

 Texto...Texto...Texto...Texto...Texto...Texto...Texto...Texto...Texto...

 Texto...Texto...Texto...Texto...Texto...Texto...Texto...Texto...Texto...

 Referencia al apendice A \autoref{sec:Apendice A}

 \subsection{Referencia cruzada} 
 Para más información, consulta la \autoref{sec:ejemplo}, que contiene detalles sobre cómo usar notas al pie 
 y referencias cruzadas.
 En esta parte vamos a poner la nota a pie de pagina\footnote{Ejemplo nota al pie de pagina}
 
 las notas tambien pueden tener referencias\footnote{\autoref{sec:ejemplo}}

 \clearpage 
% ============================
%  SECCIÓN: EJEMPLO GLOSARIOS, ABREVIACIONES, SIGLAS Y SIMBOLOS 
 %  (sección 11 en config.tex)
 %  Las entradas se cargan desde un archivo externo (17-glossary.tex)
 %  Se incluye en main.tex con: 
 %     \input{tex/03-Material_de_referencia/17-glossary.tex}
 
 \section{Ejemplo de uso de glosarios definidos}

 \textbf{Los glosarios se imprimen al final del documento*}
 Después de la primera mención de un acrónimo,
 las siguientes veces que uses la misma etiqueta mostrará la forma abreviada. 
 % ============================
 \subsection{Glosario de términos generales}
 
 % Las entradas definidas con \newglossaryentry sin "type=" van al glosario principal.
 
 En este caso la abreviacion siempre se muestra corta:
 algoritmo abreviacion = \gls{algoritmo} 
 si se define se puede usar una forma plural =  \glspl{algoritmo}  
 otro  ejemplo mal definido = \gls{modelo} 
 
 % ============================
 \subsection{Lista de abreviaciones y siglas}
 
 Las siglas definidas con newacronym se imprimen completas la primera vez, luego abreviadas.
 
 La \gls{cpu} realiza las tareas básicas de procesamiento.  
 Una \gls{gpu} moderna puede acelerar algoritmos complejos.
 Por su parte, la \gls{gpu} es más eficiente para el procesamiento gráfico intensivo.  
 
 % ============================
 \subsection{Glosario de símbolos}
  
 % Entradas definidas con type=symbols. El nombre puede tener notación matemática.
 Para usar  un glosario personalizado hay que definir el glosario en el preambulo de tu main 

 El símbolo \gls{alpha} representa el coeficiente de expansión térmica.  
 La \gls{lambda} indica la longitud de onda en fenómenos ondulatorios.
 
 \subsection{Ejemplo de uso de glosarios personalizados}
 
 La constante de Planck, representada por \gls{planck}, es fundamental en la teoría cuántica.  
 La constante de gravitación universal, denotada por \gls{gravitacional}, juega un papel crucial en la física gravitacional.

 \clearpage 
% ============================
%  SECCIÓN: EJEMPLO FIGURAS Y TABLAS
 %(seccion 9 en config.tex)

 \clearpage % Fuerza salto de página para procesar cualquier figura/tabla pendiente
 \section{Figuras y tablas}

 Este capítulo ofrece ejemplos prácticos de cómo trabajar con tablas y figuras en \LaTeX.
 
 % ----------------------------
 % SUBSECCIÓN: TABLAS
 % ----------------------------
  
  \subsection{Tablas en \LaTeX}
  
  \begin{table}[H] % [H] del paquete float obliga a poner la tabla exactamente aquí
      \centering
      \begin{tabular}{|c|c|c|} % Tres columnas centradas con bordes verticales
          \hline
          \textbf{ID} & \textbf{Nombre} & \textbf{Edad} \\
          \hline
          1 & Juan Pérez  & 30 \\
          2 & María López & 25 \\
          3 & Luis García & 35 \\
          \hline
      \end{tabular}
      \caption{Ejemplo de tabla creada con \LaTeX.}
      \label{tab:ejemplo_tabla}
  \end{table}
  
 % Usando booktabs
 \subsubsection{Usando booktabs}

 \begin{table}[H]
   \centering
   \begin{tabular}{ccc}
     \toprule
     ID & Nombre & Edad \\
     \midrule
     1 & Juan Pérez  & 30 \\
     2 & María López & 25 \\
     3 & Luis García & 35 \\
     \bottomrule
   \end{tabular}
   \caption{Tabla con estilo \texttt{booktabs}.}
   \label{tab:booktabs}
 \end{table}
 
  \subsubsection{Consejos sobre tablas}
  Si las tablas son simples, se pueden escribir directamente como en el ejemplo anterior.  
  Para tablas extensas que ocupen varias páginas, se recomienda usar el paquete `longtable` o `tabu`.
 
 % ----------------------------
 % SUBSECCIÓN: FIGURAS
 % ----------------------------
  
  \subsection{Figuras en \LaTeX}
  
  Ejemplo de cómo insertar imágenes externas:
  
  \begin{figure}[H]
      \centering
      \includegraphics[width=0.7\textwidth]{images/figura.png}
      \caption{Ejemplo de figura externa representada como una imagen.}
      \label{fig:ejemplo_figura}
  \end{figure}
  
  \subsubsection{Consejos sobre figuras}
  Para diagramas, gráficas o ilustraciones técnicas, es preferible usar herramientas como `TikZ` o `PGFPlots` en lugar de imágenes externas.  
  Esto mantiene consistencia tipográfica, escalabilidad y edición directa desde el código fuente.
  
 
\clearpage 
 
\chapter*{Epílogo}
\phantomsection % punto de referencia para hipervínculos.
\addcontentsline{toc}{chapter}{Epílogo} % Agrega el epílogo al índice
\markboth{Epílogo}{Epílogo} % Define encabezado del epílogo

% Aquí comienza el contenido del epílogo
El epílogo es una parte final donde se resumen los puntos clave o se hace una reflexión sobre el contenido presentado en el libro. Puede incluir agradecimientos adicionales, comentarios sobre el proceso de escritura o ideas para futuros proyectos.

\section*{Reflexión Final}
% Aquí puede ir una reflexión más personal o general
Aquí va tu reflexión o conclusión final sobre el tema tratado en el libro.

\section*{Agradecimientos}
% Puedes agregar un agradecimiento en el epílogo
Quiero expresar mi agradecimiento a todas las personas que han contribuido a la realización de este libro y a quienes me han apoyado a lo largo de este proyecto.

% Aquí puedes continuar con más secciones o párrafos según lo necesites.
 
%=============================================================
%=== Imprime material de referencia
%=============================================================
 \part{Material de referencia}
 %=== Apendices===
 
 \appendix
 \phantomsection
\begin{appendices}

%===========================================================
% Apendice A
%===========================================================

 \chapter{Apendices}
 \section{Apendice A}
 \label{sec:Apendice A}
 Este apéndice contiene información técnica adicional que complementa el capítulo de resultados.
 
 \subsection{Detalles técnicos del experimento}
 \subsubsection{Especificaciones del equipo}
 
 \begin{itemize}
   \item Modelo del sensor: XYZ-100
   \item Precisión: $\pm 0.01$ unidades
   \item Frecuencia de muestreo: 1 kHz
 \end{itemize}
 
 \subsubsection{Configuraciones del software}
 
 Las simulaciones fueron ejecutadas usando el entorno de desarrollo ABC en su versión 2.4.1.  
 El código fuente está disponible en el repositorio del proyecto.
 
 \subsubsection{Notas adicionales}
 
 Se realizaron pruebas preliminares para calibrar el sistema. Los resultados de estas pruebas no se incluyen en el cuerpo principal del documento por motivos de claridad.

%===========================================================
% Apendice B
%===========================================================

\section{Apendice B}
 
\end{appendices}




 
 
 %=== Glosarios (puedes simplemente comentar lo que no uses)
 
 \let\cleardoublepage\clearpage % Para evitar la hoja en blanco que se genera al terminar en hoja impar
 \printglossary[type=acronym, title={Lista de abreviaciones}]
 \let\cleardoublepage\clearpage % Para evitar la hoja en blanco que se genera al terminar en hoja impar
 \printglossary[type=symbols, title={Glosario de símbolos}]
 \let\cleardoublepage\clearpage % Para evitar la hoja en blanco que se genera al terminar en hoja impar
 \printglossary[title={Glosario de términos}]
 \let\cleardoublepage\clearpage % Para evitar la hoja en blanco que se genera al terminar en hoja impar
 \printglossary[type=constantes, title={Constantes físicas}]
 
 % Bibliografia
 %\nocite{*} % Forza a imprimir todas las entradas de la bibliografia 
 \let\cleardoublepage\clearpage % Para evitar la hoja en blanco que se genera al terminar en hoja impar
 \printbibliography 
 
 \clearpage
 \chapter*{Colofón} % Evita numeración del capítulo
\phantomsection % Punto de anclaje para hipervínculos
\addcontentsline{toc}{chapter}{Colofón} % Lo agrega al índice

Este libro fue compuesto en \LaTeX, un sistema de composición tipográfica de alta calidad.

El diseño y la maquetación fueron realizados utilizando la clase de documento previamente definida en este proyecto.

La tipografía utilizada en el cuerpo del texto es \textit{Computer Modern}
La impresión y encuadernación fueron realizadas por [nombre de la imprenta o detalles de impresión].

\vfill

% Información centrada al pie de página
\begin{center}
  \small
  \textit{Este documento fue elaborado como parte del proyecto X.} \\
  \textit{Autor: Nombre del autor.} \\
  \textit{Fecha de finalización: Abril 2025.}
  
  \vspace{1em}

  \textcopyright~
  Año, Nombre del Autor.
\end{center}

 
 
\end{document}
